% main.tex
%!TeX program = xelatex
\documentclass{NJUReport}

\usepackage{cite}
\usepackage{natbib}  % 参考文献样式支持
\usepackage{amsmath}
\usepackage{amssymb, amsthm, mathtools}
\usepackage{braket}
\usepackage{booktabs}
\usepackage{hyperref}
\usepackage{enumitem} % 用于更好的列表格式

% =============================================
% 个人信息设置
% =============================================
\headl{2025春}  % 学期
\headc{课程名称}  % 课程名称
\headr{姓名 学号}  % 姓名和学号
\lessonTitle{课程名称课程报告}  % 课程标题
\reportTitle{报告标题}  % 报告标题
\stuname{姓名}  % 学生姓名
\stuid{学号}  % 学号
\inst{学院名称}  % 学院
\major{专业名称}  % 专业
\date{\today}  % 日期

% =============================================
% 记号与环境定义(可根据需要修改)
% =============================================
\newcommand{\Tr}{\mathrm{Tr}}
\newcommand{\id}{\mathbb{I}}
\newcommand{\supp}{\mathrm{supp}}
\newcommand{\D}{\mathrm{D}}
\newcommand{\eps}{\varepsilon}

% 注意:定理、定义、命题等环境已在 NJUReport.cls 中预定义
% 如需自定义,请参考 cls 文件中的定义方式
\newtheorem{theorem}{定理}[section]

\begin{document}

% =============================================
% Part 1: 封面
% =============================================
\cover
\thispagestyle{empty}
\clearpage

% =============================================
% Part 2: 摘要
% =============================================
\begin{abstract}
这里填写报告的摘要内容。摘要应该简明扼要地概括报告的主要内容、方法、结果和结论。建议控制在300-500字以内。

摘要应包含以下要素:
\begin{itemize}
\item 研究背景和目的
\item 主要方法和理论
\item 重要发现和结论
\item 研究意义和价值
\end{itemize}

关键词可以在这里列出,用分号分隔。
\end{abstract}

\thispagestyle{empty}
\clearpage

% =============================================
% Part 3: 目录页
% =============================================
\pagenumbering{Roman}
\setcounter{page}{1}
\tableofcontents
\clearpage

% =============================================
% Part 4: 正文内容
% =============================================
\pagenumbering{arabic}
\setcounter{page}{1}

\section{引言}
在引言部分,应该介绍研究背景、研究现状、存在的问题以及本报告的目的和意义。可以包括以下内容:

\begin{itemize}
\item 研究领域的背景介绍
\item 相关工作的文献综述
\item 现有方法的局限性
\item 本报告的创新点和贡献
\item 报告的组织结构
\end{itemize}

引用文献的示例:根据Shannon的经典工作\cite{Shannon1948},信息论为现代通信奠定了理论基础。

\section{理论基础}
在这一部分介绍报告所需的理论基础和相关概念。

\subsection{基本概念}
定义相关的基本概念和术语。

\begin{theorem}[示例定理]
这里给出一个定理的示例。
\end{theorem}

\subsection{相关理论}
介绍与报告主题相关的重要理论。

\section{方法与分析}
详细描述所使用的方法、模型或算法。

\subsection{方法描述}
具体描述采用的研究方法。

\subsection{理论分析}
进行必要的理论分析和推导。

数学公式的示例:
\begin{equation}
E = mc^2
\label{eq:einstein}
\end{equation}

引用公式的示例:根据方程\eqref{eq:einstein},我们可以看到...

\section{实验与结果}
如果报告包含实验部分,在这里描述实验设计、实施过程和结果。

\subsection{实验设计}
描述实验的设计思路和具体方案。

\subsection{结果分析}
展示和分析实验结果。

可以使用表格展示数据:
\begin{table}[h]
\centering
\caption{示例表格}
\label{tab:example}
\begin{tabular}{@{}lcc@{}}
\toprule
项目 & 数值1 & 数值2 \\
\midrule
项目A & 1.23 & 4.56 \\
项目B & 2.34 & 5.67 \\
项目C & 3.45 & 6.78 \\
\bottomrule
\end{tabular}
\end{table}

\section{讨论}
对结果进行深入讨论,分析其意义和局限性。

\subsection{结果讨论}
分析结果的意义和影响。

\subsection{局限性分析}
讨论研究的局限性和不足。

\section{结论}
总结报告的主要发现和贡献,并提出未来的研究方向。

\begin{itemize}
\item 总结主要发现
\item 强调创新点和贡献
\item 指出研究局限性
\item 提出未来研究方向
\end{itemize}

% =============================================
% 参考文献
% =============================================
\bibliographystyle{alpha}
\bibliography{reference}

\end{document}
